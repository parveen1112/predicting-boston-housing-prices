
% Default to the notebook output style

    


% Inherit from the specified cell style.




    
\documentclass[11pt]{article}

    
    
    \usepackage[T1]{fontenc}
    % Nicer default font (+ math font) than Computer Modern for most use cases
    \usepackage{mathpazo}

    % Basic figure setup, for now with no caption control since it's done
    % automatically by Pandoc (which extracts ![](path) syntax from Markdown).
    \usepackage{graphicx}
    % We will generate all images so they have a width \maxwidth. This means
    % that they will get their normal width if they fit onto the page, but
    % are scaled down if they would overflow the margins.
    \makeatletter
    \def\maxwidth{\ifdim\Gin@nat@width>\linewidth\linewidth
    \else\Gin@nat@width\fi}
    \makeatother
    \let\Oldincludegraphics\includegraphics
    % Set max figure width to be 80% of text width, for now hardcoded.
    \renewcommand{\includegraphics}[1]{\Oldincludegraphics[width=.8\maxwidth]{#1}}
    % Ensure that by default, figures have no caption (until we provide a
    % proper Figure object with a Caption API and a way to capture that
    % in the conversion process - todo).
    \usepackage{caption}
    \DeclareCaptionLabelFormat{nolabel}{}
    \captionsetup{labelformat=nolabel}

    \usepackage{adjustbox} % Used to constrain images to a maximum size 
    \usepackage{xcolor} % Allow colors to be defined
    \usepackage{enumerate} % Needed for markdown enumerations to work
    \usepackage{geometry} % Used to adjust the document margins
    \usepackage{amsmath} % Equations
    \usepackage{amssymb} % Equations
    \usepackage{textcomp} % defines textquotesingle
    % Hack from http://tex.stackexchange.com/a/47451/13684:
    \AtBeginDocument{%
        \def\PYZsq{\textquotesingle}% Upright quotes in Pygmentized code
    }
    \usepackage{upquote} % Upright quotes for verbatim code
    \usepackage{eurosym} % defines \euro
    \usepackage[mathletters]{ucs} % Extended unicode (utf-8) support
    \usepackage[utf8x]{inputenc} % Allow utf-8 characters in the tex document
    \usepackage{fancyvrb} % verbatim replacement that allows latex
    \usepackage{grffile} % extends the file name processing of package graphics 
                         % to support a larger range 
    % The hyperref package gives us a pdf with properly built
    % internal navigation ('pdf bookmarks' for the table of contents,
    % internal cross-reference links, web links for URLs, etc.)
    \usepackage{hyperref}
    \usepackage{longtable} % longtable support required by pandoc >1.10
    \usepackage{booktabs}  % table support for pandoc > 1.12.2
    \usepackage[inline]{enumitem} % IRkernel/repr support (it uses the enumerate* environment)
    \usepackage[normalem]{ulem} % ulem is needed to support strikethroughs (\sout)
                                % normalem makes italics be italics, not underlines
    

    
    
    % Colors for the hyperref package
    \definecolor{urlcolor}{rgb}{0,.145,.698}
    \definecolor{linkcolor}{rgb}{.71,0.21,0.01}
    \definecolor{citecolor}{rgb}{.12,.54,.11}

    % ANSI colors
    \definecolor{ansi-black}{HTML}{3E424D}
    \definecolor{ansi-black-intense}{HTML}{282C36}
    \definecolor{ansi-red}{HTML}{E75C58}
    \definecolor{ansi-red-intense}{HTML}{B22B31}
    \definecolor{ansi-green}{HTML}{00A250}
    \definecolor{ansi-green-intense}{HTML}{007427}
    \definecolor{ansi-yellow}{HTML}{DDB62B}
    \definecolor{ansi-yellow-intense}{HTML}{B27D12}
    \definecolor{ansi-blue}{HTML}{208FFB}
    \definecolor{ansi-blue-intense}{HTML}{0065CA}
    \definecolor{ansi-magenta}{HTML}{D160C4}
    \definecolor{ansi-magenta-intense}{HTML}{A03196}
    \definecolor{ansi-cyan}{HTML}{60C6C8}
    \definecolor{ansi-cyan-intense}{HTML}{258F8F}
    \definecolor{ansi-white}{HTML}{C5C1B4}
    \definecolor{ansi-white-intense}{HTML}{A1A6B2}

    % commands and environments needed by pandoc snippets
    % extracted from the output of `pandoc -s`
    \providecommand{\tightlist}{%
      \setlength{\itemsep}{0pt}\setlength{\parskip}{0pt}}
    \DefineVerbatimEnvironment{Highlighting}{Verbatim}{commandchars=\\\{\}}
    % Add ',fontsize=\small' for more characters per line
    \newenvironment{Shaded}{}{}
    \newcommand{\KeywordTok}[1]{\textcolor[rgb]{0.00,0.44,0.13}{\textbf{{#1}}}}
    \newcommand{\DataTypeTok}[1]{\textcolor[rgb]{0.56,0.13,0.00}{{#1}}}
    \newcommand{\DecValTok}[1]{\textcolor[rgb]{0.25,0.63,0.44}{{#1}}}
    \newcommand{\BaseNTok}[1]{\textcolor[rgb]{0.25,0.63,0.44}{{#1}}}
    \newcommand{\FloatTok}[1]{\textcolor[rgb]{0.25,0.63,0.44}{{#1}}}
    \newcommand{\CharTok}[1]{\textcolor[rgb]{0.25,0.44,0.63}{{#1}}}
    \newcommand{\StringTok}[1]{\textcolor[rgb]{0.25,0.44,0.63}{{#1}}}
    \newcommand{\CommentTok}[1]{\textcolor[rgb]{0.38,0.63,0.69}{\textit{{#1}}}}
    \newcommand{\OtherTok}[1]{\textcolor[rgb]{0.00,0.44,0.13}{{#1}}}
    \newcommand{\AlertTok}[1]{\textcolor[rgb]{1.00,0.00,0.00}{\textbf{{#1}}}}
    \newcommand{\FunctionTok}[1]{\textcolor[rgb]{0.02,0.16,0.49}{{#1}}}
    \newcommand{\RegionMarkerTok}[1]{{#1}}
    \newcommand{\ErrorTok}[1]{\textcolor[rgb]{1.00,0.00,0.00}{\textbf{{#1}}}}
    \newcommand{\NormalTok}[1]{{#1}}
    
    % Additional commands for more recent versions of Pandoc
    \newcommand{\ConstantTok}[1]{\textcolor[rgb]{0.53,0.00,0.00}{{#1}}}
    \newcommand{\SpecialCharTok}[1]{\textcolor[rgb]{0.25,0.44,0.63}{{#1}}}
    \newcommand{\VerbatimStringTok}[1]{\textcolor[rgb]{0.25,0.44,0.63}{{#1}}}
    \newcommand{\SpecialStringTok}[1]{\textcolor[rgb]{0.73,0.40,0.53}{{#1}}}
    \newcommand{\ImportTok}[1]{{#1}}
    \newcommand{\DocumentationTok}[1]{\textcolor[rgb]{0.73,0.13,0.13}{\textit{{#1}}}}
    \newcommand{\AnnotationTok}[1]{\textcolor[rgb]{0.38,0.63,0.69}{\textbf{\textit{{#1}}}}}
    \newcommand{\CommentVarTok}[1]{\textcolor[rgb]{0.38,0.63,0.69}{\textbf{\textit{{#1}}}}}
    \newcommand{\VariableTok}[1]{\textcolor[rgb]{0.10,0.09,0.49}{{#1}}}
    \newcommand{\ControlFlowTok}[1]{\textcolor[rgb]{0.00,0.44,0.13}{\textbf{{#1}}}}
    \newcommand{\OperatorTok}[1]{\textcolor[rgb]{0.40,0.40,0.40}{{#1}}}
    \newcommand{\BuiltInTok}[1]{{#1}}
    \newcommand{\ExtensionTok}[1]{{#1}}
    \newcommand{\PreprocessorTok}[1]{\textcolor[rgb]{0.74,0.48,0.00}{{#1}}}
    \newcommand{\AttributeTok}[1]{\textcolor[rgb]{0.49,0.56,0.16}{{#1}}}
    \newcommand{\InformationTok}[1]{\textcolor[rgb]{0.38,0.63,0.69}{\textbf{\textit{{#1}}}}}
    \newcommand{\WarningTok}[1]{\textcolor[rgb]{0.38,0.63,0.69}{\textbf{\textit{{#1}}}}}
    
    
    % Define a nice break command that doesn't care if a line doesn't already
    % exist.
    \def\br{\hspace*{\fill} \\* }
    % Math Jax compatability definitions
    \def\gt{>}
    \def\lt{<}
    % Document parameters
    \title{boston\_housing}
    
    
    

    % Pygments definitions
    
\makeatletter
\def\PY@reset{\let\PY@it=\relax \let\PY@bf=\relax%
    \let\PY@ul=\relax \let\PY@tc=\relax%
    \let\PY@bc=\relax \let\PY@ff=\relax}
\def\PY@tok#1{\csname PY@tok@#1\endcsname}
\def\PY@toks#1+{\ifx\relax#1\empty\else%
    \PY@tok{#1}\expandafter\PY@toks\fi}
\def\PY@do#1{\PY@bc{\PY@tc{\PY@ul{%
    \PY@it{\PY@bf{\PY@ff{#1}}}}}}}
\def\PY#1#2{\PY@reset\PY@toks#1+\relax+\PY@do{#2}}

\expandafter\def\csname PY@tok@w\endcsname{\def\PY@tc##1{\textcolor[rgb]{0.73,0.73,0.73}{##1}}}
\expandafter\def\csname PY@tok@c\endcsname{\let\PY@it=\textit\def\PY@tc##1{\textcolor[rgb]{0.25,0.50,0.50}{##1}}}
\expandafter\def\csname PY@tok@cp\endcsname{\def\PY@tc##1{\textcolor[rgb]{0.74,0.48,0.00}{##1}}}
\expandafter\def\csname PY@tok@k\endcsname{\let\PY@bf=\textbf\def\PY@tc##1{\textcolor[rgb]{0.00,0.50,0.00}{##1}}}
\expandafter\def\csname PY@tok@kp\endcsname{\def\PY@tc##1{\textcolor[rgb]{0.00,0.50,0.00}{##1}}}
\expandafter\def\csname PY@tok@kt\endcsname{\def\PY@tc##1{\textcolor[rgb]{0.69,0.00,0.25}{##1}}}
\expandafter\def\csname PY@tok@o\endcsname{\def\PY@tc##1{\textcolor[rgb]{0.40,0.40,0.40}{##1}}}
\expandafter\def\csname PY@tok@ow\endcsname{\let\PY@bf=\textbf\def\PY@tc##1{\textcolor[rgb]{0.67,0.13,1.00}{##1}}}
\expandafter\def\csname PY@tok@nb\endcsname{\def\PY@tc##1{\textcolor[rgb]{0.00,0.50,0.00}{##1}}}
\expandafter\def\csname PY@tok@nf\endcsname{\def\PY@tc##1{\textcolor[rgb]{0.00,0.00,1.00}{##1}}}
\expandafter\def\csname PY@tok@nc\endcsname{\let\PY@bf=\textbf\def\PY@tc##1{\textcolor[rgb]{0.00,0.00,1.00}{##1}}}
\expandafter\def\csname PY@tok@nn\endcsname{\let\PY@bf=\textbf\def\PY@tc##1{\textcolor[rgb]{0.00,0.00,1.00}{##1}}}
\expandafter\def\csname PY@tok@ne\endcsname{\let\PY@bf=\textbf\def\PY@tc##1{\textcolor[rgb]{0.82,0.25,0.23}{##1}}}
\expandafter\def\csname PY@tok@nv\endcsname{\def\PY@tc##1{\textcolor[rgb]{0.10,0.09,0.49}{##1}}}
\expandafter\def\csname PY@tok@no\endcsname{\def\PY@tc##1{\textcolor[rgb]{0.53,0.00,0.00}{##1}}}
\expandafter\def\csname PY@tok@nl\endcsname{\def\PY@tc##1{\textcolor[rgb]{0.63,0.63,0.00}{##1}}}
\expandafter\def\csname PY@tok@ni\endcsname{\let\PY@bf=\textbf\def\PY@tc##1{\textcolor[rgb]{0.60,0.60,0.60}{##1}}}
\expandafter\def\csname PY@tok@na\endcsname{\def\PY@tc##1{\textcolor[rgb]{0.49,0.56,0.16}{##1}}}
\expandafter\def\csname PY@tok@nt\endcsname{\let\PY@bf=\textbf\def\PY@tc##1{\textcolor[rgb]{0.00,0.50,0.00}{##1}}}
\expandafter\def\csname PY@tok@nd\endcsname{\def\PY@tc##1{\textcolor[rgb]{0.67,0.13,1.00}{##1}}}
\expandafter\def\csname PY@tok@s\endcsname{\def\PY@tc##1{\textcolor[rgb]{0.73,0.13,0.13}{##1}}}
\expandafter\def\csname PY@tok@sd\endcsname{\let\PY@it=\textit\def\PY@tc##1{\textcolor[rgb]{0.73,0.13,0.13}{##1}}}
\expandafter\def\csname PY@tok@si\endcsname{\let\PY@bf=\textbf\def\PY@tc##1{\textcolor[rgb]{0.73,0.40,0.53}{##1}}}
\expandafter\def\csname PY@tok@se\endcsname{\let\PY@bf=\textbf\def\PY@tc##1{\textcolor[rgb]{0.73,0.40,0.13}{##1}}}
\expandafter\def\csname PY@tok@sr\endcsname{\def\PY@tc##1{\textcolor[rgb]{0.73,0.40,0.53}{##1}}}
\expandafter\def\csname PY@tok@ss\endcsname{\def\PY@tc##1{\textcolor[rgb]{0.10,0.09,0.49}{##1}}}
\expandafter\def\csname PY@tok@sx\endcsname{\def\PY@tc##1{\textcolor[rgb]{0.00,0.50,0.00}{##1}}}
\expandafter\def\csname PY@tok@m\endcsname{\def\PY@tc##1{\textcolor[rgb]{0.40,0.40,0.40}{##1}}}
\expandafter\def\csname PY@tok@gh\endcsname{\let\PY@bf=\textbf\def\PY@tc##1{\textcolor[rgb]{0.00,0.00,0.50}{##1}}}
\expandafter\def\csname PY@tok@gu\endcsname{\let\PY@bf=\textbf\def\PY@tc##1{\textcolor[rgb]{0.50,0.00,0.50}{##1}}}
\expandafter\def\csname PY@tok@gd\endcsname{\def\PY@tc##1{\textcolor[rgb]{0.63,0.00,0.00}{##1}}}
\expandafter\def\csname PY@tok@gi\endcsname{\def\PY@tc##1{\textcolor[rgb]{0.00,0.63,0.00}{##1}}}
\expandafter\def\csname PY@tok@gr\endcsname{\def\PY@tc##1{\textcolor[rgb]{1.00,0.00,0.00}{##1}}}
\expandafter\def\csname PY@tok@ge\endcsname{\let\PY@it=\textit}
\expandafter\def\csname PY@tok@gs\endcsname{\let\PY@bf=\textbf}
\expandafter\def\csname PY@tok@gp\endcsname{\let\PY@bf=\textbf\def\PY@tc##1{\textcolor[rgb]{0.00,0.00,0.50}{##1}}}
\expandafter\def\csname PY@tok@go\endcsname{\def\PY@tc##1{\textcolor[rgb]{0.53,0.53,0.53}{##1}}}
\expandafter\def\csname PY@tok@gt\endcsname{\def\PY@tc##1{\textcolor[rgb]{0.00,0.27,0.87}{##1}}}
\expandafter\def\csname PY@tok@err\endcsname{\def\PY@bc##1{\setlength{\fboxsep}{0pt}\fcolorbox[rgb]{1.00,0.00,0.00}{1,1,1}{\strut ##1}}}
\expandafter\def\csname PY@tok@kc\endcsname{\let\PY@bf=\textbf\def\PY@tc##1{\textcolor[rgb]{0.00,0.50,0.00}{##1}}}
\expandafter\def\csname PY@tok@kd\endcsname{\let\PY@bf=\textbf\def\PY@tc##1{\textcolor[rgb]{0.00,0.50,0.00}{##1}}}
\expandafter\def\csname PY@tok@kn\endcsname{\let\PY@bf=\textbf\def\PY@tc##1{\textcolor[rgb]{0.00,0.50,0.00}{##1}}}
\expandafter\def\csname PY@tok@kr\endcsname{\let\PY@bf=\textbf\def\PY@tc##1{\textcolor[rgb]{0.00,0.50,0.00}{##1}}}
\expandafter\def\csname PY@tok@bp\endcsname{\def\PY@tc##1{\textcolor[rgb]{0.00,0.50,0.00}{##1}}}
\expandafter\def\csname PY@tok@fm\endcsname{\def\PY@tc##1{\textcolor[rgb]{0.00,0.00,1.00}{##1}}}
\expandafter\def\csname PY@tok@vc\endcsname{\def\PY@tc##1{\textcolor[rgb]{0.10,0.09,0.49}{##1}}}
\expandafter\def\csname PY@tok@vg\endcsname{\def\PY@tc##1{\textcolor[rgb]{0.10,0.09,0.49}{##1}}}
\expandafter\def\csname PY@tok@vi\endcsname{\def\PY@tc##1{\textcolor[rgb]{0.10,0.09,0.49}{##1}}}
\expandafter\def\csname PY@tok@vm\endcsname{\def\PY@tc##1{\textcolor[rgb]{0.10,0.09,0.49}{##1}}}
\expandafter\def\csname PY@tok@sa\endcsname{\def\PY@tc##1{\textcolor[rgb]{0.73,0.13,0.13}{##1}}}
\expandafter\def\csname PY@tok@sb\endcsname{\def\PY@tc##1{\textcolor[rgb]{0.73,0.13,0.13}{##1}}}
\expandafter\def\csname PY@tok@sc\endcsname{\def\PY@tc##1{\textcolor[rgb]{0.73,0.13,0.13}{##1}}}
\expandafter\def\csname PY@tok@dl\endcsname{\def\PY@tc##1{\textcolor[rgb]{0.73,0.13,0.13}{##1}}}
\expandafter\def\csname PY@tok@s2\endcsname{\def\PY@tc##1{\textcolor[rgb]{0.73,0.13,0.13}{##1}}}
\expandafter\def\csname PY@tok@sh\endcsname{\def\PY@tc##1{\textcolor[rgb]{0.73,0.13,0.13}{##1}}}
\expandafter\def\csname PY@tok@s1\endcsname{\def\PY@tc##1{\textcolor[rgb]{0.73,0.13,0.13}{##1}}}
\expandafter\def\csname PY@tok@mb\endcsname{\def\PY@tc##1{\textcolor[rgb]{0.40,0.40,0.40}{##1}}}
\expandafter\def\csname PY@tok@mf\endcsname{\def\PY@tc##1{\textcolor[rgb]{0.40,0.40,0.40}{##1}}}
\expandafter\def\csname PY@tok@mh\endcsname{\def\PY@tc##1{\textcolor[rgb]{0.40,0.40,0.40}{##1}}}
\expandafter\def\csname PY@tok@mi\endcsname{\def\PY@tc##1{\textcolor[rgb]{0.40,0.40,0.40}{##1}}}
\expandafter\def\csname PY@tok@il\endcsname{\def\PY@tc##1{\textcolor[rgb]{0.40,0.40,0.40}{##1}}}
\expandafter\def\csname PY@tok@mo\endcsname{\def\PY@tc##1{\textcolor[rgb]{0.40,0.40,0.40}{##1}}}
\expandafter\def\csname PY@tok@ch\endcsname{\let\PY@it=\textit\def\PY@tc##1{\textcolor[rgb]{0.25,0.50,0.50}{##1}}}
\expandafter\def\csname PY@tok@cm\endcsname{\let\PY@it=\textit\def\PY@tc##1{\textcolor[rgb]{0.25,0.50,0.50}{##1}}}
\expandafter\def\csname PY@tok@cpf\endcsname{\let\PY@it=\textit\def\PY@tc##1{\textcolor[rgb]{0.25,0.50,0.50}{##1}}}
\expandafter\def\csname PY@tok@c1\endcsname{\let\PY@it=\textit\def\PY@tc##1{\textcolor[rgb]{0.25,0.50,0.50}{##1}}}
\expandafter\def\csname PY@tok@cs\endcsname{\let\PY@it=\textit\def\PY@tc##1{\textcolor[rgb]{0.25,0.50,0.50}{##1}}}

\def\PYZbs{\char`\\}
\def\PYZus{\char`\_}
\def\PYZob{\char`\{}
\def\PYZcb{\char`\}}
\def\PYZca{\char`\^}
\def\PYZam{\char`\&}
\def\PYZlt{\char`\<}
\def\PYZgt{\char`\>}
\def\PYZsh{\char`\#}
\def\PYZpc{\char`\%}
\def\PYZdl{\char`\$}
\def\PYZhy{\char`\-}
\def\PYZsq{\char`\'}
\def\PYZdq{\char`\"}
\def\PYZti{\char`\~}
% for compatibility with earlier versions
\def\PYZat{@}
\def\PYZlb{[}
\def\PYZrb{]}
\makeatother


    % Exact colors from NB
    \definecolor{incolor}{rgb}{0.0, 0.0, 0.5}
    \definecolor{outcolor}{rgb}{0.545, 0.0, 0.0}



    
    % Prevent overflowing lines due to hard-to-break entities
    \sloppy 
    % Setup hyperref package
    \hypersetup{
      breaklinks=true,  % so long urls are correctly broken across lines
      colorlinks=true,
      urlcolor=urlcolor,
      linkcolor=linkcolor,
      citecolor=citecolor,
      }
    % Slightly bigger margins than the latex defaults
    
    \geometry{verbose,tmargin=1in,bmargin=1in,lmargin=1in,rmargin=1in}
    
    

    \begin{document}
    
    
    \maketitle
    
    

    
    \section{Machine Learning Engineer
Nanodegree}\label{machine-learning-engineer-nanodegree}

\subsection{Model Evaluation \&
Validation}\label{model-evaluation-validation}

\subsection{Project: Predicting Boston Housing
Prices}\label{project-predicting-boston-housing-prices}

Welcome to the first project of the Machine Learning Engineer
Nanodegree! In this notebook, some template code has already been
provided for you, and you will need to implement additional
functionality to successfully complete this project. You will not need
to modify the included code beyond what is requested. Sections that
begin with \textbf{'Implementation'} in the header indicate that the
following block of code will require additional functionality which you
must provide. Instructions will be provided for each section and the
specifics of the implementation are marked in the code block with a
'TODO' statement. Please be sure to read the instructions carefully!

In addition to implementing code, there will be questions that you must
answer which relate to the project and your implementation. Each section
where you will answer a question is preceded by a \textbf{'Question X'}
header. Carefully read each question and provide thorough answers in the
following text boxes that begin with \textbf{'Answer:'}. Your project
submission will be evaluated based on your answers to each of the
questions and the implementation you provide.

\begin{quote}
\textbf{Note:} Code and Markdown cells can be executed using the
\textbf{Shift + Enter} keyboard shortcut. In addition, Markdown cells
can be edited by typically double-clicking the cell to enter edit mode.
\end{quote}

    \subsection{Getting Started}\label{getting-started}

In this project, you will evaluate the performance and predictive power
of a model that has been trained and tested on data collected from homes
in suburbs of Boston, Massachusetts. A model trained on this data that
is seen as a \emph{good fit} could then be used to make certain
predictions about a home --- in particular, its monetary value. This
model would prove to be invaluable for someone like a real estate agent
who could make use of such information on a daily basis.

The dataset for this project originates from the
\href{https://archive.ics.uci.edu/ml/datasets/Housing}{UCI Machine
Learning Repository}. The Boston housing data was collected in 1978 and
each of the 506 entries represent aggregated data about 14 features for
homes from various suburbs in Boston, Massachusetts. For the purposes of
this project, the following preprocessing steps have been made to the
dataset: - 16 data points have an
\texttt{\textquotesingle{}MEDV\textquotesingle{}} value of 50.0. These
data points likely contain \textbf{missing or censored values} and have
been removed. - 1 data point has an
\texttt{\textquotesingle{}RM\textquotesingle{}} value of 8.78. This data
point can be considered an \textbf{outlier} and has been removed. - The
features \texttt{\textquotesingle{}RM\textquotesingle{}},
\texttt{\textquotesingle{}LSTAT\textquotesingle{}},
\texttt{\textquotesingle{}PTRATIO\textquotesingle{}}, and
\texttt{\textquotesingle{}MEDV\textquotesingle{}} are essential. The
remaining \textbf{non-relevant features} have been excluded. - The
feature \texttt{\textquotesingle{}MEDV\textquotesingle{}} has been
\textbf{multiplicatively scaled} to account for 35 years of market
inflation.

Run the code cell below to load the Boston housing dataset, along with a
few of the necessary Python libraries required for this project. You
will know the dataset loaded successfully if the size of the dataset is
reported.

    \begin{Verbatim}[commandchars=\\\{\}]
{\color{incolor}In [{\color{incolor}24}]:} \PY{c+c1}{\PYZsh{} Import libraries necessary for this project}
         \PY{k+kn}{import} \PY{n+nn}{numpy} \PY{k}{as} \PY{n+nn}{np}
         \PY{k+kn}{import} \PY{n+nn}{pandas} \PY{k}{as} \PY{n+nn}{pd}
         \PY{k+kn}{from} \PY{n+nn}{sklearn}\PY{n+nn}{.}\PY{n+nn}{cross\PYZus{}validation} \PY{k}{import} \PY{n}{ShuffleSplit}
         
         \PY{c+c1}{\PYZsh{} Import supplementary visualizations code visuals.py}
         \PY{k+kn}{import} \PY{n+nn}{visuals} \PY{k}{as} \PY{n+nn}{vs}
         
         \PY{c+c1}{\PYZsh{} Pretty display for notebooks}
         \PY{o}{\PYZpc{}}\PY{k}{matplotlib} inline
         
         \PY{c+c1}{\PYZsh{} Load the Boston housing dataset}
         \PY{n}{data} \PY{o}{=} \PY{n}{pd}\PY{o}{.}\PY{n}{read\PYZus{}csv}\PY{p}{(}\PY{l+s+s1}{\PYZsq{}}\PY{l+s+s1}{housing.csv}\PY{l+s+s1}{\PYZsq{}}\PY{p}{)}
         \PY{n}{prices} \PY{o}{=} \PY{n}{data}\PY{p}{[}\PY{l+s+s1}{\PYZsq{}}\PY{l+s+s1}{MEDV}\PY{l+s+s1}{\PYZsq{}}\PY{p}{]}
         \PY{n}{features} \PY{o}{=} \PY{n}{data}\PY{o}{.}\PY{n}{drop}\PY{p}{(}\PY{l+s+s1}{\PYZsq{}}\PY{l+s+s1}{MEDV}\PY{l+s+s1}{\PYZsq{}}\PY{p}{,} \PY{n}{axis} \PY{o}{=} \PY{l+m+mi}{1}\PY{p}{)}
             
         \PY{c+c1}{\PYZsh{} Success}
         \PY{n+nb}{print}\PY{p}{(}\PY{l+s+s2}{\PYZdq{}}\PY{l+s+s2}{Boston housing dataset has }\PY{l+s+si}{\PYZob{}\PYZcb{}}\PY{l+s+s2}{ data points with }\PY{l+s+si}{\PYZob{}\PYZcb{}}\PY{l+s+s2}{ variables each.}\PY{l+s+s2}{\PYZdq{}}\PY{o}{.}\PY{n}{format}\PY{p}{(}\PY{o}{*}\PY{n}{data}\PY{o}{.}\PY{n}{shape}\PY{p}{)}\PY{p}{)}
\end{Verbatim}


    \begin{Verbatim}[commandchars=\\\{\}]
Boston housing dataset has 489 data points with 4 variables each.

    \end{Verbatim}

    \subsection{Data Exploration}\label{data-exploration}

In this first section of this project, you will make a cursory
investigation about the Boston housing data and provide your
observations. Familiarizing yourself with the data through an
explorative process is a fundamental practice to help you better
understand and justify your results.

Since the main goal of this project is to construct a working model
which has the capability of predicting the value of houses, we will need
to separate the dataset into \textbf{features} and the \textbf{target
variable}. The \textbf{features},
\texttt{\textquotesingle{}RM\textquotesingle{}},
\texttt{\textquotesingle{}LSTAT\textquotesingle{}}, and
\texttt{\textquotesingle{}PTRATIO\textquotesingle{}}, give us
quantitative information about each data point. The \textbf{target
variable}, \texttt{\textquotesingle{}MEDV\textquotesingle{}}, will be
the variable we seek to predict. These are stored in \texttt{features}
and \texttt{prices}, respectively.

    \subsubsection{Implementation: Calculate
Statistics}\label{implementation-calculate-statistics}

For your very first coding implementation, you will calculate
descriptive statistics about the Boston housing prices. Since
\texttt{numpy} has already been imported for you, use this library to
perform the necessary calculations. These statistics will be extremely
important later on to analyze various prediction results from the
constructed model.

In the code cell below, you will need to implement the following: -
Calculate the minimum, maximum, mean, median, and standard deviation of
\texttt{\textquotesingle{}MEDV\textquotesingle{}}, which is stored in
\texttt{prices}. - Store each calculation in their respective variable.

    \begin{Verbatim}[commandchars=\\\{\}]
{\color{incolor}In [{\color{incolor}25}]:} \PY{c+c1}{\PYZsh{} TODO: Minimum price of the data}
         \PY{n}{minimum\PYZus{}price} \PY{o}{=} \PY{n}{np}\PY{o}{.}\PY{n}{min}\PY{p}{(}\PY{n}{prices}\PY{p}{)}
         
         \PY{c+c1}{\PYZsh{} TODO: Maximum price of the data}
         \PY{n}{maximum\PYZus{}price} \PY{o}{=} \PY{n}{np}\PY{o}{.}\PY{n}{max}\PY{p}{(}\PY{n}{prices}\PY{p}{)}
         
         \PY{c+c1}{\PYZsh{} TODO: Mean price of the data}
         \PY{n}{mean\PYZus{}price} \PY{o}{=} \PY{n}{np}\PY{o}{.}\PY{n}{mean}\PY{p}{(}\PY{n}{prices}\PY{p}{)}
         
         \PY{c+c1}{\PYZsh{} TODO: Median price of the data}
         \PY{n}{median\PYZus{}price} \PY{o}{=} \PY{n}{np}\PY{o}{.}\PY{n}{median}\PY{p}{(}\PY{n}{prices}\PY{p}{)}
         
         \PY{c+c1}{\PYZsh{} TODO: Standard deviation of prices of the data}
         \PY{n}{std\PYZus{}price} \PY{o}{=} \PY{n}{np}\PY{o}{.}\PY{n}{std}\PY{p}{(}\PY{n}{prices}\PY{p}{)}
         
         \PY{c+c1}{\PYZsh{} Show the calculated statistics}
         \PY{n+nb}{print}\PY{p}{(}\PY{l+s+s2}{\PYZdq{}}\PY{l+s+s2}{Statistics for Boston housing dataset:}\PY{l+s+se}{\PYZbs{}n}\PY{l+s+s2}{\PYZdq{}}\PY{p}{)}
         \PY{n+nb}{print}\PY{p}{(}\PY{l+s+s2}{\PYZdq{}}\PY{l+s+s2}{Minimum price: \PYZdl{}}\PY{l+s+si}{\PYZob{}\PYZcb{}}\PY{l+s+s2}{\PYZdq{}}\PY{o}{.}\PY{n}{format}\PY{p}{(}\PY{n}{minimum\PYZus{}price}\PY{p}{)}\PY{p}{)} 
         \PY{n+nb}{print}\PY{p}{(}\PY{l+s+s2}{\PYZdq{}}\PY{l+s+s2}{Maximum price: \PYZdl{}}\PY{l+s+si}{\PYZob{}\PYZcb{}}\PY{l+s+s2}{\PYZdq{}}\PY{o}{.}\PY{n}{format}\PY{p}{(}\PY{n}{maximum\PYZus{}price}\PY{p}{)}\PY{p}{)}
         \PY{n+nb}{print}\PY{p}{(}\PY{l+s+s2}{\PYZdq{}}\PY{l+s+s2}{Mean price: \PYZdl{}}\PY{l+s+si}{\PYZob{}\PYZcb{}}\PY{l+s+s2}{\PYZdq{}}\PY{o}{.}\PY{n}{format}\PY{p}{(}\PY{n}{mean\PYZus{}price}\PY{p}{)}\PY{p}{)}
         \PY{n+nb}{print}\PY{p}{(}\PY{l+s+s2}{\PYZdq{}}\PY{l+s+s2}{Median price \PYZdl{}}\PY{l+s+si}{\PYZob{}\PYZcb{}}\PY{l+s+s2}{\PYZdq{}}\PY{o}{.}\PY{n}{format}\PY{p}{(}\PY{n}{median\PYZus{}price}\PY{p}{)}\PY{p}{)}
         \PY{n+nb}{print}\PY{p}{(}\PY{l+s+s2}{\PYZdq{}}\PY{l+s+s2}{Standard deviation of prices: \PYZdl{}}\PY{l+s+si}{\PYZob{}\PYZcb{}}\PY{l+s+s2}{\PYZdq{}}\PY{o}{.}\PY{n}{format}\PY{p}{(}\PY{n}{std\PYZus{}price}\PY{p}{)}\PY{p}{)}
\end{Verbatim}


    \begin{Verbatim}[commandchars=\\\{\}]
Statistics for Boston housing dataset:

Minimum price: \$105000.0
Maximum price: \$1024800.0
Mean price: \$454342.9447852761
Median price \$438900.0
Standard deviation of prices: \$165171.13154429477

    \end{Verbatim}

    \subsubsection{Question 1 - Feature
Observation}\label{question-1---feature-observation}

As a reminder, we are using three features from the Boston housing
dataset: \texttt{\textquotesingle{}RM\textquotesingle{}},
\texttt{\textquotesingle{}LSTAT\textquotesingle{}}, and
\texttt{\textquotesingle{}PTRATIO\textquotesingle{}}. For each data
point (neighborhood): - \texttt{\textquotesingle{}RM\textquotesingle{}}
is the average number of rooms among homes in the neighborhood. -
\texttt{\textquotesingle{}LSTAT\textquotesingle{}} is the percentage of
homeowners in the neighborhood considered "lower class" (working poor).
- \texttt{\textquotesingle{}PTRATIO\textquotesingle{}} is the ratio of
students to teachers in primary and secondary schools in the
neighborhood.

** Using your intuition, for each of the three features above, do you
think that an increase in the value of that feature would lead to an
\textbf{increase} in the value of
\texttt{\textquotesingle{}MEDV\textquotesingle{}} or a \textbf{decrease}
in the value of \texttt{\textquotesingle{}MEDV\textquotesingle{}}?
Justify your answer for each.**

\textbf{Hint:} This problem can phrased using examples like below.\\
* Would you expect a home that has an
\texttt{\textquotesingle{}RM\textquotesingle{}} value(number of rooms)
of 6 be worth more or less than a home that has an
\texttt{\textquotesingle{}RM\textquotesingle{}} value of 7? * Would you
expect a neighborhood that has an
\texttt{\textquotesingle{}LSTAT\textquotesingle{}} value(percent of
lower class workers) of 15 have home prices be worth more or less than a
neighborhood that has an
\texttt{\textquotesingle{}LSTAT\textquotesingle{}} value of 20? * Would
you expect a neighborhood that has an
\texttt{\textquotesingle{}PTRATIO\textquotesingle{}} value(ratio of
students to teachers) of 10 have home prices be worth more or less than
a neighborhood that has an
\texttt{\textquotesingle{}PTRATIO\textquotesingle{}} value of 15?

    \textbf{Answer: }

\texttt{Would\ you\ expect\ a\ home\ that\ has\ an\ \textquotesingle{}RM\textquotesingle{}\ value(number\ of\ rooms)\ of\ 6\ be\ worth\ more\ or\ less\ than\ a\ home\ that\ has\ an\ \textquotesingle{}RM\textquotesingle{}\ value\ of\ 7?}

RM values signifies Number of rooms and more space means more price. So
RM value of 7 would be more worth.

\textbf{RM increases, MEDV increases. }

\texttt{Would\ you\ expect\ a\ neighborhood\ that\ has\ an\ \textquotesingle{}LSTAT\textquotesingle{}\ value(percent\ of\ lower\ class\ workers)\ of\ 15\ have\ home\ prices\ be\ worth\ more\ or\ less\ than\ a\ neighborhood\ that\ has\ an\ \textquotesingle{}LSTAT\textquotesingle{}\ value\ of\ 20?}

LSTAT signifies \% of lower class homeowners. If \% of lower class
homeowners would be more that means prices won't be high. Lower class
owners won't be financially strong and would likely to buy lower rates
houses. More \% means prices are lesser in that area. So, LSTAT value of
20 have home prices be of less worth than of 15.

\textbf{LSTAT increase, MEDV decreases. }

\texttt{Would\ you\ expect\ a\ neighborhood\ that\ has\ an\ \textquotesingle{}PTRATIO\textquotesingle{}\ value(ratio\ of\ students\ to\ teachers)\ of\ 10\ have\ home\ prices\ be\ worth\ more\ or\ less\ than\ a\ neighborhood\ that\ has\ an\ \textquotesingle{}PTRATIO\textquotesingle{}\ value\ of\ 15?}

PTRATIO signifies students/teachers. If its high that means teachers are
lesser than students. 1 teacher is teaching a large group of students
which signifies shortage of teachers. Teachers are more likely to be
high in neighborhood which is good and pay them well which corresponds
to financially strong people in neighborhood. If the teachers are less
which means neighbourbood is not good or lower class owners. So,
'PTRATIO' value of 15 have home prices be of lesser worth than 15.

\textbf{PTRATIO increase, MEDV decreases. }

According to the intuition:

\begin{enumerate}
\def\labelenumi{\arabic{enumi}.}
\tightlist
\item
  RM increases, MEDV increases.
\item
  LSTAT increase, MEDV decreases.
\item
  PTRATIO increase, MEDV decreases.
\end{enumerate}

    \begin{center}\rule{0.5\linewidth}{\linethickness}\end{center}

\subsection{Developing a Model}\label{developing-a-model}

In this second section of the project, you will develop the tools and
techniques necessary for a model to make a prediction. Being able to
make accurate evaluations of each model's performance through the use of
these tools and techniques helps to greatly reinforce the confidence in
your predictions.

    \subsubsection{Implementation: Define a Performance
Metric}\label{implementation-define-a-performance-metric}

It is difficult to measure the quality of a given model without
quantifying its performance over training and testing. This is typically
done using some type of performance metric, whether it is through
calculating some type of error, the goodness of fit, or some other
useful measurement. For this project, you will be calculating the
\href{http://stattrek.com/statistics/dictionary.aspx?definition=coefficient_of_determination}{\emph{coefficient
of determination}}, R2, to quantify your model's performance. The
coefficient of determination for a model is a useful statistic in
regression analysis, as it often describes how "good" that model is at
making predictions.

The values for R2 range from 0 to 1, which captures the percentage of
squared correlation between the predicted and actual values of the
\textbf{target variable}. A model with an R2 of 0 is no better than a
model that always predicts the \emph{mean} of the target variable,
whereas a model with an R2 of 1 perfectly predicts the target variable.
Any value between 0 and 1 indicates what percentage of the target
variable, using this model, can be explained by the \textbf{features}.
\emph{A model can be given a negative R2 as well, which indicates that
the model is \textbf{arbitrarily worse} than one that always predicts
the mean of the target variable.}

For the \texttt{performance\_metric} function in the code cell below,
you will need to implement the following: - Use \texttt{r2\_score} from
\texttt{sklearn.metrics} to perform a performance calculation between
\texttt{y\_true} and \texttt{y\_predict}. - Assign the performance score
to the \texttt{score} variable.

    \begin{Verbatim}[commandchars=\\\{\}]
{\color{incolor}In [{\color{incolor}26}]:} \PY{c+c1}{\PYZsh{} TODO: Import \PYZsq{}r2\PYZus{}score\PYZsq{}}
         \PY{k+kn}{from} \PY{n+nn}{sklearn}\PY{n+nn}{.}\PY{n+nn}{metrics} \PY{k}{import} \PY{n}{r2\PYZus{}score}
         
         \PY{k}{def} \PY{n+nf}{performance\PYZus{}metric}\PY{p}{(}\PY{n}{y\PYZus{}true}\PY{p}{,} \PY{n}{y\PYZus{}predict}\PY{p}{)}\PY{p}{:}
             \PY{l+s+sd}{\PYZdq{}\PYZdq{}\PYZdq{} Calculates and returns the performance score between }
         \PY{l+s+sd}{        true and predicted values based on the metric chosen. \PYZdq{}\PYZdq{}\PYZdq{}}
             
             \PY{c+c1}{\PYZsh{} TODO: Calculate the performance score between \PYZsq{}y\PYZus{}true\PYZsq{} and \PYZsq{}y\PYZus{}predict\PYZsq{}}
             \PY{n}{score} \PY{o}{=} \PY{n}{r2\PYZus{}score}\PY{p}{(}\PY{n}{y\PYZus{}true}\PY{p}{,} \PY{n}{y\PYZus{}predict}\PY{p}{)}
             
             \PY{c+c1}{\PYZsh{} Return the score}
             \PY{k}{return} \PY{n}{score}
\end{Verbatim}


    \subsubsection{Question 2 - Goodness of
Fit}\label{question-2---goodness-of-fit}

Assume that a dataset contains five data points and a model made the
following predictions for the target variable:

\begin{longtable}[]{@{}cc@{}}
\toprule
True Value & Prediction\tabularnewline
\midrule
\endhead
3.0 & 2.5\tabularnewline
-0.5 & 0.0\tabularnewline
2.0 & 2.1\tabularnewline
7.0 & 7.8\tabularnewline
4.2 & 5.3\tabularnewline
\bottomrule
\end{longtable}

Run the code cell below to use the \texttt{performance\_metric} function
and calculate this model's coefficient of determination.

    \begin{Verbatim}[commandchars=\\\{\}]
{\color{incolor}In [{\color{incolor}27}]:} \PY{c+c1}{\PYZsh{} Calculate the performance of this model}
         \PY{n}{score} \PY{o}{=} \PY{n}{performance\PYZus{}metric}\PY{p}{(}\PY{p}{[}\PY{l+m+mi}{3}\PY{p}{,} \PY{o}{\PYZhy{}}\PY{l+m+mf}{0.5}\PY{p}{,} \PY{l+m+mi}{2}\PY{p}{,} \PY{l+m+mi}{7}\PY{p}{,} \PY{l+m+mf}{4.2}\PY{p}{]}\PY{p}{,} \PY{p}{[}\PY{l+m+mf}{2.5}\PY{p}{,} \PY{l+m+mf}{0.0}\PY{p}{,} \PY{l+m+mf}{2.1}\PY{p}{,} \PY{l+m+mf}{7.8}\PY{p}{,} \PY{l+m+mf}{5.3}\PY{p}{]}\PY{p}{)}
         \PY{n+nb}{print}\PY{p}{(}\PY{l+s+s2}{\PYZdq{}}\PY{l+s+s2}{Model has a coefficient of determination, R\PYZca{}2, of }\PY{l+s+si}{\PYZob{}:.3f\PYZcb{}}\PY{l+s+s2}{.}\PY{l+s+s2}{\PYZdq{}}\PY{o}{.}\PY{n}{format}\PY{p}{(}\PY{n}{score}\PY{p}{)}\PY{p}{)}
\end{Verbatim}


    \begin{Verbatim}[commandchars=\\\{\}]
Model has a coefficient of determination, R\^{}2, of 0.923.

    \end{Verbatim}

    \begin{itemize}
\tightlist
\item
  Would you consider this model to have successfully captured the
  variation of the target variable?
\item
  Why or why not?
\end{itemize}

** Hint: ** The R2 score is the proportion of the variance in the
dependent variable that is predictable from the independent variable. In
other words: * R2 score of 0 means that the dependent variable cannot be
predicted from the independent variable. * R2 score of 1 means the
dependent variable can be predicted from the independent variable. * R2
score between 0 and 1 indicates the extent to which the dependent
variable is predictable. An * R2 score of 0.40 means that 40 percent of
the variance in Y is predictable from X.

    \textbf{Answer:}

\textbf{Coefficient of Determination i.e. R2} : It captures the
percentage of squared correlation between the predicted and actual
values of the target variable. A model with an R2 of 0 is no better than
a model that always predicts the mean of the target variable, whereas a
model with an R2 of 1 perfectly predicts the target variable.

As a definition indicates, value closest to 1 i.e. \texttt{0.923} will
be a good value for capturing the variation of the target variable. So,
It is.

    \subsubsection{Implementation: Shuffle and Split
Data}\label{implementation-shuffle-and-split-data}

Your next implementation requires that you take the Boston housing
dataset and split the data into training and testing subsets. Typically,
the data is also shuffled into a random order when creating the training
and testing subsets to remove any bias in the ordering of the dataset.

For the code cell below, you will need to implement the following: - Use
\texttt{train\_test\_split} from \texttt{sklearn.cross\_validation} to
shuffle and split the \texttt{features} and \texttt{prices} data into
training and testing sets. - Split the data into 80\% training and 20\%
testing. - Set the \texttt{random\_state} for
\texttt{train\_test\_split} to a value of your choice. This ensures
results are consistent. - Assign the train and testing splits to
\texttt{X\_train}, \texttt{X\_test}, \texttt{y\_train}, and
\texttt{y\_test}.

    \begin{Verbatim}[commandchars=\\\{\}]
{\color{incolor}In [{\color{incolor}28}]:} \PY{c+c1}{\PYZsh{} TODO: Import \PYZsq{}train\PYZus{}test\PYZus{}split\PYZsq{}}
         
         \PY{k+kn}{from} \PY{n+nn}{sklearn}\PY{n+nn}{.}\PY{n+nn}{cross\PYZus{}validation} \PY{k}{import} \PY{n}{train\PYZus{}test\PYZus{}split}
         
         \PY{c+c1}{\PYZsh{} TODO: Shuffle and split the data into training and testing subsets}
         
         \PY{c+c1}{\PYZsh{}shuffle=True was not working as it is supported in 0.19 and above}
         \PY{n}{X\PYZus{}train}\PY{p}{,} \PY{n}{X\PYZus{}test}\PY{p}{,} \PY{n}{y\PYZus{}train}\PY{p}{,} \PY{n}{y\PYZus{}test} \PY{o}{=} \PY{n}{train\PYZus{}test\PYZus{}split}\PY{p}{(}\PY{n}{features}\PY{p}{,} \PY{n}{prices}\PY{p}{,} \PY{n}{test\PYZus{}size}\PY{o}{=}\PY{l+m+mf}{0.2}\PY{p}{,} \PY{n}{random\PYZus{}state}\PY{o}{=}\PY{l+m+mi}{50}\PY{p}{)}
         
         \PY{c+c1}{\PYZsh{} Success}
         \PY{n+nb}{print}\PY{p}{(}\PY{l+s+s2}{\PYZdq{}}\PY{l+s+s2}{Training and testing split was successful.}\PY{l+s+s2}{\PYZdq{}}\PY{p}{)}
\end{Verbatim}


    \begin{Verbatim}[commandchars=\\\{\}]
Training and testing split was successful.

    \end{Verbatim}

    \subsubsection{Question 3 - Training and
Testing}\label{question-3---training-and-testing}

\begin{itemize}
\tightlist
\item
  What is the benefit to splitting a dataset into some ratio of training
  and testing subsets for a learning algorithm?
\end{itemize}

\textbf{Hint:} Think about how overfitting or underfitting is contingent
upon how splits on data is done.

    \textbf{Answer: }

Splitting the dataset into ratio of training and testing subsets is a
measuring tool which helps in measuring the accuracy of the model. For
this, testing subset should not see the light of the day unless until
training is finished.

Overfitting occurs when we tries to perfectly fit the training set which
leads to inacurrate predictions. (When we train the model with more
training set) Underfitting occurs when we simplifies a complex model
based the training set which leads to inaccurate predictions. (Less
training set can lead to underfitting)

\textbf{Detecting Overfitting}: If our model works good on training set
than on testing set, then we are likely overfitting.

\textbf{Detecting Underfitting}: If our model works poor on training set
as well as testing set.

    \begin{center}\rule{0.5\linewidth}{\linethickness}\end{center}

\subsection{Analyzing Model
Performance}\label{analyzing-model-performance}

In this third section of the project, you'll take a look at several
models' learning and testing performances on various subsets of training
data. Additionally, you'll investigate one particular algorithm with an
increasing \texttt{\textquotesingle{}max\_depth\textquotesingle{}}
parameter on the full training set to observe how model complexity
affects performance. Graphing your model's performance based on varying
criteria can be beneficial in the analysis process, such as visualizing
behavior that may not have been apparent from the results alone.

    \subsubsection{Learning Curves}\label{learning-curves}

The following code cell produces four graphs for a decision tree model
with different maximum depths. Each graph visualizes the learning curves
of the model for both training and testing as the size of the training
set is increased. Note that the shaded region of a learning curve
denotes the uncertainty of that curve (measured as the standard
deviation). The model is scored on both the training and testing sets
using R2, the coefficient of determination.

Run the code cell below and use these graphs to answer the following
question.

    \begin{Verbatim}[commandchars=\\\{\}]
{\color{incolor}In [{\color{incolor}29}]:} \PY{c+c1}{\PYZsh{} Produce learning curves for varying training set sizes and maximum depths}
         \PY{n}{vs}\PY{o}{.}\PY{n}{ModelLearning}\PY{p}{(}\PY{n}{features}\PY{p}{,} \PY{n}{prices}\PY{p}{)}
\end{Verbatim}


    \begin{center}
    \adjustimage{max size={0.9\linewidth}{0.9\paperheight}}{output_21_0.png}
    \end{center}
    { \hspace*{\fill} \\}
    
    \subsubsection{Question 4 - Learning the
Data}\label{question-4---learning-the-data}

\begin{itemize}
\tightlist
\item
  Choose one of the graphs above and state the maximum depth for the
  model.
\item
  What happens to the score of the training curve as more training
  points are added? What about the testing curve?
\item
  Would having more training points benefit the model?
\end{itemize}

\textbf{Hint:} Are the learning curves converging to particular scores?
Generally speaking, the more data you have, the better. But if your
training and testing curves are converging with a score above your
benchmark threshold, would this be necessary? Think about the pros and
cons of adding more training points based on if the training and testing
curves are converging.

    \textbf{Answer: }

I would choose the learning curve with max\_depth : 3. As it can be
seen, training and testing are appearing to collide at a higher score.
As training set is increased, score decreased, but after 300 points, it
appears to be somewhat straight. As testing set is increased, score
increased, but after 300 points, it appears to be somewhat straight.
Adding more training points will lead both the curve to converge at some
point, but after that adding more training points is useless.

    \subsubsection{Complexity Curves}\label{complexity-curves}

The following code cell produces a graph for a decision tree model that
has been trained and validated on the training data using different
maximum depths. The graph produces two complexity curves --- one for
training and one for validation. Similar to the \textbf{learning
curves}, the shaded regions of both the complexity curves denote the
uncertainty in those curves, and the model is scored on both the
training and validation sets using the \texttt{performance\_metric}
function.

** Run the code cell below and use this graph to answer the following
two questions Q5 and Q6. **

    \begin{Verbatim}[commandchars=\\\{\}]
{\color{incolor}In [{\color{incolor}30}]:} \PY{n}{vs}\PY{o}{.}\PY{n}{ModelComplexity}\PY{p}{(}\PY{n}{X\PYZus{}train}\PY{p}{,} \PY{n}{y\PYZus{}train}\PY{p}{)}
\end{Verbatim}


    \begin{center}
    \adjustimage{max size={0.9\linewidth}{0.9\paperheight}}{output_25_0.png}
    \end{center}
    { \hspace*{\fill} \\}
    
    \subsubsection{Question 5 - Bias-Variance
Tradeoff}\label{question-5---bias-variance-tradeoff}

\begin{itemize}
\tightlist
\item
  When the model is trained with a maximum depth of 1, does the model
  suffer from high bias or from high variance?
\item
  How about when the model is trained with a maximum depth of 10? What
  visual cues in the graph justify your conclusions?
\end{itemize}

\textbf{Hint:} High bias is a sign of underfitting(model is not complex
enough to pick up the nuances in the data) and high variance is a sign
of overfitting(model is by-hearting the data and cannot generalize
well). Think about which model(depth 1 or 10) aligns with which part of
the tradeoff.

    \textbf{Answer: }

By looking at the graph, one can see training score and testing score
corresponding to depth 1 is around 0.4 which implies lower training
score and lower testing score which corresponds to underfitting i.e.
High Bias.

When the model is trained at the depth of 10, it has High Variance i.e.
Overfitting. Visual Cues: 1. Gap between the Training and Validation
Score. 2. Consistent Variance in the validation score. 3. Overfitting
works well with training set but not with testing set as it can seen
with their scores in the graph.

    \subsubsection{Question 6 - Best-Guess Optimal
Model}\label{question-6---best-guess-optimal-model}

\begin{itemize}
\tightlist
\item
  Which maximum depth do you think results in a model that best
  generalizes to unseen data?
\item
  What intuition lead you to this answer?
\end{itemize}

** Hint: ** Look at the graph above Question 5 and see where the
validation scores lie for the various depths that have been assigned to
the model. Does it get better with increased depth? At what point do we
get our best validation score without overcomplicating our model? And
remember, Occams Razor states "Among competing hypotheses, the one with
the fewest assumptions should be selected."

    \textbf{Answer: }

Best depth would be 4 because after that depth, both the curve starts
diverging from each other and validation score goes down. Increasing the
depth decreases the validation score from this depth, validation score
is highest at depth 4 and training score is also good.

    \begin{center}\rule{0.5\linewidth}{\linethickness}\end{center}

\subsection{Evaluating Model
Performance}\label{evaluating-model-performance}

In this final section of the project, you will construct a model and
make a prediction on the client's feature set using an optimized model
from \texttt{fit\_model}.

    \subsubsection{Question 7 - Grid Search}\label{question-7---grid-search}

\begin{itemize}
\tightlist
\item
  What is the grid search technique?
\item
  How it can be applied to optimize a learning algorithm?
\end{itemize}

** Hint: ** When explaining the Grid Search technique, be sure to touch
upon why it is used, what the 'grid' entails and what the end goal of
this method is. To solidify your answer, you can also give an example of
a parameter in a model that can be optimized using this approach.

    \textbf{Answer: }

Grid-search is the process of scanning the data to configure optimal
parameters for a given model. Depending on the model, some of the
parameters are necessary. Grid Search will build a model on each
parameter combination possible and iterates through every parameter
combination and stores a score for each combination.

For Example:

In case of decision trees, you have the following parameters, criterion
- \texttt{gini} or \texttt{entropy}, splitter - \texttt{best} or
\texttt{random} and max\_depth - \texttt{3} or \texttt{4}.

Grid search will create all the combination of all the three parameters
and find the best score with the optimal values of these parameters.

    \subsubsection{Question 8 -
Cross-Validation}\label{question-8---cross-validation}

\begin{itemize}
\item
  What is the k-fold cross-validation training technique?
\item
  What benefit does this technique provide for grid search when
  optimizing a model?
\end{itemize}

\textbf{Hint:} When explaining the k-fold cross validation technique, be
sure to touch upon what 'k' is, how the dataset is split into different
parts for training and testing and the number of times it is run based
on the 'k' value.

When thinking about how k-fold cross validation helps grid search, think
about the main drawbacks of grid search which are hinged upon
\textbf{using a particular subset of data for training or testing} and
how k-fold cv could help alleviate that. You can refer to the
\href{http://scikit-learn.org/stable/modules/cross_validation.html\#cross-validation}{docs}
for your answer.

    \textbf{Answer: }

K-fold Cross validation training is a cross validation technique in
which a dataset is divided into k equal subset which each subset is
known as fold. Folds be f1,f2....fk. Keep fi to be the validation set
and rest of the (k -1) set to be the cross validation set. Train your
model using the cross validation set and calculate the accuracy using
the validation set. This needs to be done for each fold. Actual accuracy
would be the average of all the accuracies obtained against each fold.

Benefit of using K Fold technique for grid search when optimizing a
model is that K fold technique consider each point exactly once in test
set and k -1 in training set which leads to less variance and better
results. Grid search choose the combination of parameters which results
in good optimal score. Here, we use the validation set many times and
this can lead to overifitting of the model. With cross validation, it
can be avoided, as our validation set becomes larger and different.

    \subsubsection{Implementation: Fitting a
Model}\label{implementation-fitting-a-model}

Your final implementation requires that you bring everything together
and train a model using the \textbf{decision tree algorithm}. To ensure
that you are producing an optimized model, you will train the model
using the grid search technique to optimize the
\texttt{\textquotesingle{}max\_depth\textquotesingle{}} parameter for
the decision tree. The
\texttt{\textquotesingle{}max\_depth\textquotesingle{}} parameter can be
thought of as how many questions the decision tree algorithm is allowed
to ask about the data before making a prediction. Decision trees are
part of a class of algorithms called \emph{supervised learning
algorithms}.

In addition, you will find your implementation is using
\texttt{ShuffleSplit()} for an alternative form of cross-validation (see
the \texttt{\textquotesingle{}cv\_sets\textquotesingle{}} variable).
While it is not the K-Fold cross-validation technique you describe in
\textbf{Question 8}, this type of cross-validation technique is just as
useful!. The \texttt{ShuffleSplit()} implementation below will create 10
(\texttt{\textquotesingle{}n\_splits\textquotesingle{}}) shuffled sets,
and for each shuffle, 20\%
(\texttt{\textquotesingle{}test\_size\textquotesingle{}}) of the data
will be used as the \emph{validation set}. While you're working on your
implementation, think about the contrasts and similarities it has to the
K-fold cross-validation technique.

Please note that ShuffleSplit has different parameters in scikit-learn
versions 0.17 and 0.18. For the \texttt{fit\_model} function in the code
cell below, you will need to implement the following: - Use
\href{http://scikit-learn.org/stable/modules/generated/sklearn.tree.DecisionTreeRegressor.html}{\texttt{DecisionTreeRegressor}}
from \texttt{sklearn.tree} to create a decision tree regressor object. -
Assign this object to the
\texttt{\textquotesingle{}regressor\textquotesingle{}} variable. -
Create a dictionary for
\texttt{\textquotesingle{}max\_depth\textquotesingle{}} with the values
from 1 to 10, and assign this to the
\texttt{\textquotesingle{}params\textquotesingle{}} variable. - Use
\href{http://scikit-learn.org/stable/modules/generated/sklearn.metrics.make_scorer.html}{\texttt{make\_scorer}}
from \texttt{sklearn.metrics} to create a scoring function object. -
Pass the \texttt{performance\_metric} function as a parameter to the
object. - Assign this scoring function to the
\texttt{\textquotesingle{}scoring\_fnc\textquotesingle{}} variable. -
Use
\href{http://scikit-learn.org/0.17/modules/generated/sklearn.grid_search.GridSearchCV.html}{\texttt{GridSearchCV}}
from \texttt{sklearn.grid\_search} to create a grid search object. -
Pass the variables
\texttt{\textquotesingle{}regressor\textquotesingle{}},
\texttt{\textquotesingle{}params\textquotesingle{}},
\texttt{\textquotesingle{}scoring\_fnc\textquotesingle{}}, and
\texttt{\textquotesingle{}cv\_sets\textquotesingle{}} as parameters to
the object. - Assign the \texttt{GridSearchCV} object to the
\texttt{\textquotesingle{}grid\textquotesingle{}} variable.

    \begin{Verbatim}[commandchars=\\\{\}]
{\color{incolor}In [{\color{incolor}31}]:} \PY{c+c1}{\PYZsh{} TODO: Import \PYZsq{}make\PYZus{}scorer\PYZsq{}, \PYZsq{}DecisionTreeRegressor\PYZsq{}, and \PYZsq{}GridSearchCV\PYZsq{}}
         \PY{k+kn}{from} \PY{n+nn}{sklearn}\PY{n+nn}{.}\PY{n+nn}{metrics} \PY{k}{import} \PY{n}{make\PYZus{}scorer}
         \PY{k+kn}{from} \PY{n+nn}{sklearn}\PY{n+nn}{.}\PY{n+nn}{tree} \PY{k}{import} \PY{n}{DecisionTreeRegressor}
         \PY{k+kn}{from} \PY{n+nn}{sklearn}\PY{n+nn}{.}\PY{n+nn}{grid\PYZus{}search} \PY{k}{import} \PY{n}{GridSearchCV}
         
         \PY{k}{def} \PY{n+nf}{fit\PYZus{}model}\PY{p}{(}\PY{n}{X}\PY{p}{,} \PY{n}{y}\PY{p}{)}\PY{p}{:}
             \PY{l+s+sd}{\PYZdq{}\PYZdq{}\PYZdq{} Performs grid search over the \PYZsq{}max\PYZus{}depth\PYZsq{} parameter for a }
         \PY{l+s+sd}{        decision tree regressor trained on the input data [X, y]. \PYZdq{}\PYZdq{}\PYZdq{}}
             
             \PY{c+c1}{\PYZsh{} Create cross\PYZhy{}validation sets from the training data}
             \PY{c+c1}{\PYZsh{} sklearn version 0.18: ShuffleSplit(n\PYZus{}splits=10, test\PYZus{}size=0.1, train\PYZus{}size=None, random\PYZus{}state=None)}
             \PY{c+c1}{\PYZsh{} sklearn versiin 0.17: ShuffleSplit(n, n\PYZus{}iter=10, test\PYZus{}size=0.1, train\PYZus{}size=None, random\PYZus{}state=None)}
             \PY{n}{cv\PYZus{}sets} \PY{o}{=} \PY{n}{ShuffleSplit}\PY{p}{(}\PY{n}{X}\PY{o}{.}\PY{n}{shape}\PY{p}{[}\PY{l+m+mi}{0}\PY{p}{]}\PY{p}{,} \PY{n}{n\PYZus{}iter} \PY{o}{=} \PY{l+m+mi}{10}\PY{p}{,} \PY{n}{test\PYZus{}size} \PY{o}{=} \PY{l+m+mf}{0.20}\PY{p}{,} \PY{n}{random\PYZus{}state} \PY{o}{=} \PY{l+m+mi}{0}\PY{p}{)}
         
             \PY{c+c1}{\PYZsh{} TODO: Create a decision tree regressor object}
             \PY{n}{regressor} \PY{o}{=} \PY{n}{DecisionTreeRegressor}\PY{p}{(}\PY{p}{)}
         
             \PY{c+c1}{\PYZsh{} TODO: Create a dictionary for the parameter \PYZsq{}max\PYZus{}depth\PYZsq{} with a range from 1 to 10}
             \PY{n}{params} \PY{o}{=} \PY{p}{\PYZob{}}\PY{l+s+s1}{\PYZsq{}}\PY{l+s+s1}{max\PYZus{}depth}\PY{l+s+s1}{\PYZsq{}}\PY{p}{:} \PY{n+nb}{list}\PY{p}{(}\PY{n+nb}{range}\PY{p}{(}\PY{l+m+mi}{1}\PY{p}{,}\PY{l+m+mi}{11}\PY{p}{)}\PY{p}{)}\PY{p}{\PYZcb{}}
         
             \PY{c+c1}{\PYZsh{} TODO: Transform \PYZsq{}performance\PYZus{}metric\PYZsq{} into a scoring function using \PYZsq{}make\PYZus{}scorer\PYZsq{} }
             \PY{n}{scoring\PYZus{}fnc} \PY{o}{=} \PY{n}{make\PYZus{}scorer}\PY{p}{(}\PY{n}{performance\PYZus{}metric}\PY{p}{)}
         
             \PY{c+c1}{\PYZsh{} TODO: Create the grid search cv object \PYZhy{}\PYZhy{}\PYZgt{} GridSearchCV()}
             \PY{c+c1}{\PYZsh{} Make sure to include the right parameters in the object:}
             \PY{c+c1}{\PYZsh{} (estimator, param\PYZus{}grid, scoring, cv) which have values \PYZsq{}regressor\PYZsq{}, \PYZsq{}params\PYZsq{}, \PYZsq{}scoring\PYZus{}fnc\PYZsq{}, and \PYZsq{}cv\PYZus{}sets\PYZsq{} respectively.}
             \PY{n}{grid} \PY{o}{=} \PY{n}{GridSearchCV}\PY{p}{(}\PY{n}{regressor}\PY{p}{,} \PY{n}{param\PYZus{}grid}\PY{o}{=}\PY{n}{params}\PY{p}{,} \PY{n}{scoring}\PY{o}{=}\PY{n}{scoring\PYZus{}fnc}\PY{p}{,} \PY{n}{cv}\PY{o}{=}\PY{n}{cv\PYZus{}sets}\PY{p}{)}
         
             \PY{c+c1}{\PYZsh{} Fit the grid search object to the data to compute the optimal model}
             \PY{n}{grid} \PY{o}{=} \PY{n}{grid}\PY{o}{.}\PY{n}{fit}\PY{p}{(}\PY{n}{X}\PY{p}{,} \PY{n}{y}\PY{p}{)}
         
             \PY{c+c1}{\PYZsh{} Return the optimal model after fitting the data}
             \PY{k}{return} \PY{n}{grid}\PY{o}{.}\PY{n}{best\PYZus{}estimator\PYZus{}}
\end{Verbatim}


    \subsubsection{Making Predictions}\label{making-predictions}

Once a model has been trained on a given set of data, it can now be used
to make predictions on new sets of input data. In the case of a
\emph{decision tree regressor}, the model has learned \emph{what the
best questions to ask about the input data are}, and can respond with a
prediction for the \textbf{target variable}. You can use these
predictions to gain information about data where the value of the target
variable is unknown --- such as data the model was not trained on.

    \subsubsection{Question 9 - Optimal
Model}\label{question-9---optimal-model}

\begin{itemize}
\tightlist
\item
  What maximum depth does the optimal model have? How does this result
  compare to your guess in \textbf{Question 6}?
\end{itemize}

Run the code block below to fit the decision tree regressor to the
training data and produce an optimal model.

    \begin{Verbatim}[commandchars=\\\{\}]
{\color{incolor}In [{\color{incolor}32}]:} \PY{c+c1}{\PYZsh{} Fit the training data to the model using grid search}
         \PY{n}{reg} \PY{o}{=} \PY{n}{fit\PYZus{}model}\PY{p}{(}\PY{n}{X\PYZus{}train}\PY{p}{,} \PY{n}{y\PYZus{}train}\PY{p}{)}
         
         \PY{c+c1}{\PYZsh{} Produce the value for \PYZsq{}max\PYZus{}depth\PYZsq{}}
         \PY{n+nb}{print}\PY{p}{(}\PY{l+s+s2}{\PYZdq{}}\PY{l+s+s2}{Parameter }\PY{l+s+s2}{\PYZsq{}}\PY{l+s+s2}{max\PYZus{}depth}\PY{l+s+s2}{\PYZsq{}}\PY{l+s+s2}{ is }\PY{l+s+si}{\PYZob{}\PYZcb{}}\PY{l+s+s2}{ for the optimal model.}\PY{l+s+s2}{\PYZdq{}}\PY{o}{.}\PY{n}{format}\PY{p}{(}\PY{n}{reg}\PY{o}{.}\PY{n}{get\PYZus{}params}\PY{p}{(}\PY{p}{)}\PY{p}{[}\PY{l+s+s1}{\PYZsq{}}\PY{l+s+s1}{max\PYZus{}depth}\PY{l+s+s1}{\PYZsq{}}\PY{p}{]}\PY{p}{)}\PY{p}{)}
\end{Verbatim}


    \begin{Verbatim}[commandchars=\\\{\}]
Parameter 'max\_depth' is 4 for the optimal model.

    \end{Verbatim}

    ** Hint: ** The answer comes from the output of the code snipped above.

\textbf{Answer: } Optimal depth is 4 which is same as that of my answer.
I made a pretty good guess.

    \subsubsection{Question 10 - Predicting Selling
Prices}\label{question-10---predicting-selling-prices}

Imagine that you were a real estate agent in the Boston area looking to
use this model to help price homes owned by your clients that they wish
to sell. You have collected the following information from three of your
clients:

\begin{longtable}[]{@{}cccc@{}}
\toprule
Feature & Client 1 & Client 2 & Client 3\tabularnewline
\midrule
\endhead
Total number of rooms in home & 5 rooms & 4 rooms & 8
rooms\tabularnewline
Neighborhood poverty level (as \%) & 17\% & 32\% & 3\%\tabularnewline
Student-teacher ratio of nearby schools & 15-to-1 & 22-to-1 &
12-to-1\tabularnewline
\bottomrule
\end{longtable}

\begin{itemize}
\tightlist
\item
  What price would you recommend each client sell his/her home at?
\item
  Do these prices seem reasonable given the values for the respective
  features?
\end{itemize}

\textbf{Hint:} Use the statistics you calculated in the \textbf{Data
Exploration} section to help justify your response. Of the three
clients, client 3 has has the biggest house, in the best public school
neighborhood with the lowest poverty level; while client 2 has the
smallest house, in a neighborhood with a relatively high poverty rate
and not the best public schools.

Run the code block below to have your optimized model make predictions
for each client's home.

    \begin{Verbatim}[commandchars=\\\{\}]
{\color{incolor}In [{\color{incolor}33}]:} \PY{c+c1}{\PYZsh{} Produce a matrix for client data}
         \PY{n}{client\PYZus{}data} \PY{o}{=} \PY{p}{[}\PY{p}{[}\PY{l+m+mi}{5}\PY{p}{,} \PY{l+m+mi}{17}\PY{p}{,} \PY{l+m+mi}{15}\PY{p}{]}\PY{p}{,} \PY{c+c1}{\PYZsh{} Client 1}
                        \PY{p}{[}\PY{l+m+mi}{4}\PY{p}{,} \PY{l+m+mi}{32}\PY{p}{,} \PY{l+m+mi}{22}\PY{p}{]}\PY{p}{,} \PY{c+c1}{\PYZsh{} Client 2}
                        \PY{p}{[}\PY{l+m+mi}{8}\PY{p}{,} \PY{l+m+mi}{3}\PY{p}{,} \PY{l+m+mi}{12}\PY{p}{]}\PY{p}{]}  \PY{c+c1}{\PYZsh{} Client 3}
         
         \PY{c+c1}{\PYZsh{} Show predictions}
         \PY{k}{for} \PY{n}{i}\PY{p}{,} \PY{n}{price} \PY{o+ow}{in} \PY{n+nb}{enumerate}\PY{p}{(}\PY{n}{reg}\PY{o}{.}\PY{n}{predict}\PY{p}{(}\PY{n}{client\PYZus{}data}\PY{p}{)}\PY{p}{)}\PY{p}{:}
             \PY{n+nb}{print}\PY{p}{(}\PY{l+s+s2}{\PYZdq{}}\PY{l+s+s2}{Predicted selling price for Client }\PY{l+s+si}{\PYZob{}\PYZcb{}}\PY{l+s+s2}{\PYZsq{}}\PY{l+s+s2}{s home: \PYZdl{}}\PY{l+s+si}{\PYZob{}:,.2f\PYZcb{}}\PY{l+s+s2}{\PYZdq{}}\PY{o}{.}\PY{n}{format}\PY{p}{(}\PY{n}{i}\PY{o}{+}\PY{l+m+mi}{1}\PY{p}{,} \PY{n}{price}\PY{p}{)}\PY{p}{)}
\end{Verbatim}


    \begin{Verbatim}[commandchars=\\\{\}]
Predicted selling price for Client 1's home: \$404,911.11
Predicted selling price for Client 2's home: \$212,223.53
Predicted selling price for Client 3's home: \$938,053.85

    \end{Verbatim}

    \textbf{Answer: }

Client 1 - Price is near the mean price having 5 rooms and medium
poverty level and student teacher ratio is ok, its a good sell.

Client 2 - Low rooms, hight poverty levels and poor Student teacher
ratio and minimum price is around 100K, I guess its a good sell.

Client 3 - Large number of rooms, Low poverty level and Good Student
teacher ratio with price near to max selling price, its a good sell.

    \subsubsection{Sensitivity}\label{sensitivity}

An optimal model is not necessarily a robust model. Sometimes, a model
is either too complex or too simple to sufficiently generalize to new
data. Sometimes, a model could use a learning algorithm that is not
appropriate for the structure of the data given. Other times, the data
itself could be too noisy or contain too few samples to allow a model to
adequately capture the target variable --- i.e., the model is
underfitted.

\textbf{Run the code cell below to run the \texttt{fit\_model} function
ten times with different training and testing sets to see how the
prediction for a specific client changes with respect to the data it's
trained on.}

    \begin{Verbatim}[commandchars=\\\{\}]
{\color{incolor}In [{\color{incolor}34}]:} \PY{n}{vs}\PY{o}{.}\PY{n}{PredictTrials}\PY{p}{(}\PY{n}{features}\PY{p}{,} \PY{n}{prices}\PY{p}{,} \PY{n}{fit\PYZus{}model}\PY{p}{,} \PY{n}{client\PYZus{}data}\PY{p}{)}
\end{Verbatim}


    \begin{Verbatim}[commandchars=\\\{\}]
Trial 1: \$391,183.33
Trial 2: \$419,700.00
Trial 3: \$415,800.00
Trial 4: \$420,622.22
Trial 5: \$413,334.78
Trial 6: \$411,931.58
Trial 7: \$399,663.16
Trial 8: \$407,232.00
Trial 9: \$351,577.61
Trial 10: \$413,700.00

Range in prices: \$69,044.61

    \end{Verbatim}

    \subsubsection{Question 11 -
Applicability}\label{question-11---applicability}

\begin{itemize}
\tightlist
\item
  In a few sentences, discuss whether the constructed model should or
  should not be used in a real-world setting.
\end{itemize}

\textbf{Hint:} Take a look at the range in prices as calculated in the
code snippet above. Some questions to answering: - How relevant today is
data that was collected from 1978? How important is inflation? - Are the
features present in the data sufficient to describe a home? Do you think
factors like quality of apppliances in the home, square feet of the plot
area, presence of pool or not etc should factor in? - Is the model
robust enough to make consistent predictions? - Would data collected in
an urban city like Boston be applicable in a rural city? - Is it fair to
judge the price of an individual home based on the characteristics of
the entire neighborhood?

    \textbf{Answer: }

\begin{enumerate}
\def\labelenumi{\arabic{enumi}.}
\tightlist
\item
  Data of 1978 is not relevant today. Inflation is directly proportional
  to price of any good.
\item
  Features present in not sufficient to describe a home, there are many
  other parameters as well. Yes,quality of apppliances in the home,
  square feet of the plot area, presence of pool or not etc also affect
  the prices. If a house has pool will be much costlier as its
  maintenance and system is costly.
\item
  Model is not robust enough to make predictions as its sensitive. For a
  specific client, model is giving results which is having a difference
  of 70K.
\item
  Data collected in urban city like Bostan can not be applicable on
  rural city as each neighborhood has different measuring parameters.
\item
  Yes, it if fair to judge the price of an individual home based on
  characteristics of entire neighbourhood as it plays an important role.
  Depending neighborhood safety, people etc affects the house prices.
\end{enumerate}

    \begin{quote}
\textbf{Note}: Once you have completed all of the code implementations
and successfully answered each question above, you may finalize your
work by exporting the iPython Notebook as an HTML document. You can do
this by using the menu above and navigating to\\
\textbf{File -\textgreater{} Download as -\textgreater{} HTML (.html)}.
Include the finished document along with this notebook as your
submission.
\end{quote}


    % Add a bibliography block to the postdoc
    
    
    
    \end{document}
